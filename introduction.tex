
Die Schnelle Fourier-Transformation (eng. \textit{fast fourier-transform (FFT)})ist eine der wichtigsten Operationen in der Signalverarbeitung. Mit ihr kann ein durch Amplitude angegebenes Signal in den Frequenzbereich transformiert werden. Dabei werden die enthaltenen Frequenzen im Ursprungssignal erkennbar.\\
Sie findet daher Anwendung in Wissenschaft und Industrie in Bereichen wie Natursimulation, Maschinellem Lernen, Fabrikation, Ingenieurswesen oder Signalmodulation.\\
Die Umsetzung im speziellen der 3D-FFT auf hochparallelen Grafikprozessoren kann enorme Beschleunigungen in der Berechnungszeit erbringen.
Durch die effiziente Umsetzung im Speziellen in einer Multi-Prozessor-Umgebung sollen gr"o"sere Inputdimensionen performant erschlossen werden. Durch die GPU-Beschleunigung und den verwendeten Systemen treten jedoch Problemstellungen auf, zu denen Lösungsansätze vorgestellt werden sollen.\\

